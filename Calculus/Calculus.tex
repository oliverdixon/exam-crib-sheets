% Calculus Examination Crib Sheet
% INCOMPLETE: SEE TODO MARKERS
%
% OWD 2023

% Examination Crib Sheet Common Preamble
% OWD 2023

\documentclass[10pt]{article}
\usepackage[a4paper, margin=2em, landscape]{geometry}
\usepackage{array, xcolor, fancyhdr, amssymb, mathtools}
\usepackage[en-GB]{datetime2}
\usepackage[
        colorlinks = true,
        allcolors = darkgray
]{hyperref}

\title{\modulename\ Examination Crib Sheet (2022/23)}
\author{Oliver Dixon}

\setlength{\parindent}{0pt}

% Header height set-up, in accordance with the margins
\renewcommand{\headrulewidth}{0pt}
\setlength{\headheight}{18pt}
\setlength{\headsep}{1em}
\addtolength{\topmargin}{\headheight}
\makeatletter
\addtolength{\topmargin}{\Gm@tmargin}
\makeatother

% Footer height set-up
\renewcommand{\footrulewidth}{0pt}
\addtolength{\textheight}{-6em}

% Header and footer body set-up
\pagestyle{fancy}
\makeatletter
\fancyhf{}
\fancyhead[L]{\large\textbf{\@title}}
\fancyhead[R]{\large Compiled by \@author\ on \today}
\makeatother
\fancyfoot[L]{\large\textsc{Side \thepage}}
\fancyfoot[R]{\large\url{https://github.com/oliverdixon/exam-crib-sheets/tree/%
        master/\modulename}}

\renewcommand{\arraystretch}{1.2}
\newcommand{\mathconst}[1]{\ensuremath\mathrm{#1}}
\DeclareMathOperator{\sgn}{sgn}


\newcommand{\modulename}{Calculus}

\begin{document}
%
% SIDE 1
%
\centering
\begin{tabular}{|m{.31\linewidth}|m{.31\linewidth}|m{.31\linewidth}|}
        \hline
        %
        \textbf{L'H{\^ o}pital's Rule}: If $f$ and $g$ are differentiable
        functions at $x_0$, $f(x_0)=g_(x_0)=0$, and $g^\prime(x_0)\neq 0$, then
        $\lim_{x\to x_0} f(x)/g(x)=\lim_{x\to x_0} f^\prime(x)/g^\prime(x)$. &
        %
        \textbf{The IVT}: Suppose $a<b$ and $f$ is continuous on $[a,b]$.
        Then, for every $y$ such that $\min(f(a),f(b)) < y < \max(f(a),f(b))$,
        there exist $x_0\in (a,b)$ s.t.\ $f(x_0)=y$. &
        %
        \textbf{The Chain Rule}: If $g$ is differentiable at $x$ and $f$ is
        differentiable at $g(x)$, then $f\circ g$ is differentiable at $x$, and
        $(f\circ g)^\prime(x)=f^\prime(g(x))g^\prime(x)$. \\
        %
        \hline
        %
        \textbf{The IFT}: If $f:I\to\mathbb{R}$ is continuous and strictly
        monotonic, then $f^{-1}:J\to I$ is also continuous, where $J=f(I)$ and
        $f^{-1}(f(x))=x$ and $f(f^{-1}(y))=y$. &
        %
        \textbf{The MVT}: If $f:[a,b]\to\mathbb{R}$ is continuous and
        differentiable on $(a,b)$, then there exist $x_0\in (a,b)$ such that
        $f^\prime(x_0)=\left[f(b)-f(a)\right]/(b-a)$. &
        %
        \textbf{Classifying CPs}: If $f:[a,b]\to\mathbb{R}$, $f^\prime$,
        $f^{\prime\prime}$ are sensibly defined, and $x_0\in (a,b)$ s.t.\ %
        $f^\prime(x_0)=0$, then $f^{\prime\prime}(x_0)>0$ means local min., and
        $f^{\prime\prime}(x_0)<0$ means local max. \\
        %
        \hline
        %
        \textbf{Taylor's Theorem (1)}: If $f\in C^{N+1}(I)$ and $x\in I$, then
        $f(x)=\sum_{n=0}^N \left[f^{(n)}(x_0)(x-x_0)^n\right]/n!+1/N!
        \int_{x_0}^x (x-t)^N f^{(N+1)}(t)\mathrm{d}t$. &
        %
        \textbf{Taylor's Theorem (2)}: The terms under the summation are the
        \emph{Taylor polynomial} of $f$ at $x_0$, of order $N$. The integral
        term is known as the \emph{error in integral form}. &
        %
        \textbf{Taylor's Theorem (3)}: The Lagrange form of the error is
        $R_N(x)=\left[(x-x_0)^{N+1}f^{(N+1)(c)}\right]/(N+1)!$, for some $c$
        between $x_0$ and $x$. \\
        %
        \hline
\end{tabular}
\clearpage
%
% SIDE 2
%
\begin{tabular}{|m{.31\linewidth}|m{.31\linewidth}|m{.31\linewidth}|}
        \hline
        1 & 2 & 3 \\
        \hline
        4 & 5 & 6 \\
        \hline
\end{tabular}
\end{document}


% Algebra Examination Crib Sheet
% OWD 2023

\input{../preamble.tex}
\newcommand{\modulename}{Algebra}
\newcommand{\norm}[1]{\ensuremath\|#1\|\,}

\begin{document}
\centering
\begin{tabular}{|m{.31\linewidth}|m{.31\linewidth}|m{.31\linewidth}|}
\hline

\textbf{De Moivre's Theorem}:
    For $ w, z \in \mathbb{C} $ such that $ w = \rho
    \mathconst{e}^{\mathconst{i} \varphi} $ and $ z = r
    \mathconst{e}^{\mathconst{i} \theta} $,
    $ wz = \rho r \mathconst{e}^{\mathconst{i}(\varphi + \theta)} $.
    Alternatively, for $ x \in \mathbb{R} $ and $ n \in \mathbb{Z} $,
    $ \cos(nx) + \mathconst{i}\sin(nx) = (\cos x + \mathconst{i} \sin x)^n $. &

\textbf{Complex Sine}:
    For $ z \in \mathbb{C} $,
    \smash{$
        \sin z = \dfrac{1}{2 \mathconst{i}} (\mathconst{e}^{\mathconst{i} z} -
        \mathconst{e}^{-\mathconst{i} z}).
    $} &

\textbf{Complex Cosine}:
    For $ z \in \mathbb{C} $,
    \smash{$
        \cos z = \dfrac{1}{2} (\mathconst{e}^{\mathconst{i} z} +
        \mathconst{e}^{-\mathconst{i} z}).
    $} \\

\hline

\textbf{Roots of Unity (Computing)}:
    There are exactly $ n $ $ n^{\text{th}} $ roots of unity. In general,
    $ \zeta^{(n)}_1 = \mathconst{e}^0 = 1 $,
    $ \zeta^{(n)}_2 = \mathconst{e}^{2\pi\mathconst{i}/n} $, \ldots,
    $ \zeta^{(n)}_3 = \mathconst{e}^{4\pi\mathconst{i}/n} $, \ldots,
    $ \zeta^{(n)}_n = \mathconst{e}^{2(n-1)\pi\mathconst{i}/n} $. &

\textbf{Roots of Unity (Properties)}:
    For $ n \geq 2 $, $ \sum_{j=1}^n \zeta^{(n)}_j = 0 $ and $ \prod_{j=1}^n
    \zeta^{(n)}_j = (-1)^{n+1} $. All roots also lie on the unit circle, such
    that $ \vert \zeta^{(n)}_j \vert = 1 $ for $ 1 \leq j \leq n $. &

\textbf{Euclid's Algorithm (1)}:
    Given $ a \in \mathbb{Z} $ and $ b \in \mathbb{N} $, there exist $ q_1 $ and
    $ r_1 $ such that $ a = q_1 b + r_1 $ and $ 0 \leq r_1 < b $. If $ r_1 = 0
    $, then $ \gcd(a, b) = b $. Otherwise, \ldots \\

\hline

\textbf{Euclid's Algorithm (2)}:
    \ldots\ find $ q_2 $ and $ r_2 $ such that $ b = q_2 r_1 + r_2 $ and $ 0
    \leq r_2 < r_1 $. If $ r_2 = 0 $, set $ \gcd(b, r_1) = \gcd(a, b) = r_1 $.
    Otherwise, iterate on the remainders. &

\textbf{Solving Congruence Equations (1)}:
    To solve $ ax \equiv b \pmod{m} $, find $ d = \gcd(a, m) $. If $ d \nmid
    b $, there are no solutions. Otherwise, write $ b = b^\prime d $ and
    $ d = sa + tm $. &

\textbf{Solving Congruence Equations (2)}:
    A single solution is $ x = sb^\prime $; let $ m^\prime = m / d $. All
    numbers congruent to $ sb^\prime \pmod{m^\prime} $ comprise the full
    solution set. \\

\hline

\textbf{Chinese Remainder Theorem}:
    If $ m_1 $ and $ m_2 $ are s.t.\ $ \gcd(m_1, m_2) = 1 $, then there is a
    unique $ x \pmod{m_1m_2} $ that satisfies $ x \equiv c_1 \pmod{m_1} $ and $
    x \equiv c_2 \pmod{m_2} $. &

\textbf{Binomial Theorem}:
    \smash{$\displaystyle
        (x+y)^n = \sum_{j=0}^n \binom{n}{j}\,x^{n-j}
    $} &

\textbf{Binomial Coefficient}:
    \smash{$\displaystyle
        \binom{n}{j} = \frac{n!}{j!(n-j)!}
    $} \\

\hline

\textbf{Cross Product}:
    \smash{$
        \begin{pmatrix}
            u_1 \\ u_2 \\ u_3
        \end{pmatrix} \times
        \begin{pmatrix}
            v_1 \\ v_2 \\ v_3
        \end{pmatrix} =
        \det\begin{pmatrix}
            \vec{i} & \vec{j} & \vec{k} \\
            u_1 & u_2 & u_3 \\
            v_1 & v_2 & v_3
        \end{pmatrix}
    $} &

\textbf{STP}:
    \smash{$
        \begin{pmatrix}
            t_1 \\ t_2 \\ t_3
        \end{pmatrix} \cdot
        \left[\begin{pmatrix}
            u_1 \\ u_2 \\ u_3
        \end{pmatrix} \times
        \begin{pmatrix}
            v_1 \\ v_2 \\ v_3
        \end{pmatrix}\right] =
        \det\begin{pmatrix}
            t_1 & t_2 & t_3 \\
            u_1 & u_2 & u_3 \\
            v_1 & v_2 & v_3
        \end{pmatrix}
    $} &

\textbf{Parametric Line}:
    An equation of a line can be written as $ x = p_1 + tv_1 $,
    $ y = p_2 + tv_2 $, and $ z = p_3 + tv_3 $, for $ t \in \mathbb{R} $.
    $ (p_1, p_2, p_3) $ are the coordinates of a point on the line, and
    $ (v_1, v_2, v_3) $ denotes the direction of the line. \\

\hline

\textbf{Point-to-Line}:
    If $ \vec{v} $ is a vector parallel to the line, $ \vec{p} $ is the position
    vector of a point on the line, and $ \vec{r} $ is a point, then the minimal
    distance from $ R $ to the line is $ \vert (\vec{p} - \vec{r}) \times
    \vec{v} \vert/\vert\vec{v}\vert $. &

\textbf{Plane Equations}:
    The position vector of an arbitrary point $ X $ in the plane is
    $ \vec{x} = \vec{p} + s \vec{u} + t \vec{v} $ for $ s, t \in \mathbb{R} $,
    where $ \vec{u} $ and $ \vec{v} $ are parallel to the plane, and $ P $ is a
    plane point. &

\textbf{Intersection of Line and Plane}:
    Write the line and plane in parametric form, equate each component, and
    solve for $ s $, $ r $, and $ t $. \\

\hline

\textbf{Normal Vector to a Plane}:
    If $ \vec{p} $ is the position vector of a known point, and $ \vec{x} $ is
    the position vector of an arbitrary point, then $ ( \vec{x} - \vec{p} )
    \cdot \vec{n} = 0 $, where $ \vec{n} $ is the normal vector. &

\textbf{Point-to-Plane}:
    The distance from a point $ A $ to a plane is
    $ \vert (\vec{p} - \vec{a}) \cdot \vec{n} \vert / \vert \vec{n} \vert $,
    where $ \vec{n} $ is the normal vector, and $ P $ is a known point in the
    plane. &

\textbf{Matrix Multiplication}:
    If $ A \in \mathbb{F}^{m \times n} $ and $ B \in \mathbb{F}^{n \times p} $,
    then the product $ AB $ is such that $ (AB)_{ik} = \sum_{j=1}^n A_{ij}
    B_{jk} $. (For our purposes, the field $ \mathbb{F} $ is the set of reals
    $ \mathbb{R} $.) \\

\hline

\textbf{Diagonal and Transpose Matrices}:
    A square matrix $ D \in \mathbb{F}^{n \times n} $ is \emph{diagonal} if
    $ D_{ij} = 0 $ for all $ i \neq j $. Diagonal matrices commute under
    multiplication. Also, for $ B \in \mathbb{F}^{m \times n}$, $ A^T_{ij} =
    A_{ji} $ for $ 1 \leq i \leq n $ and $ 1 \leq j \leq m $. &

\textbf{Echelon Forms (1 and 2)}:
    \smash{$
        \begin{pmatrix}
            \star & \square & \square \\
            0 & \star & \square \\
            0 & 0 & \star
        \end{pmatrix}
        \begin{pmatrix}
            \star & \square & \square \\
            0 & \star & \square \\
            0 & 0 & 0
        \end{pmatrix}
    $} &

\textbf{Echelon Forms (3 and 4)}:
    \smash{$
        \begin{pmatrix}
            \star & \square & \square \\
            0 & 0 & \star \\
            0 & 0 & 0
        \end{pmatrix}
        \begin{pmatrix}
            \star & \square & \square \\
            0 & 0 & 0 \\
            0 & 0 & 0
        \end{pmatrix}
    $} \\

\hline

\textbf{Reduced EFs}:
    \smash{$
        \begin{pmatrix}
            \star & 0 & 0 \\
            0 & \star & 0 \\
            0 & 0 & \star
        \end{pmatrix}
        \begin{pmatrix}
            \star & 0 & \square \\
            0 & \star & \square \\
            0 & 0 & \star
        \end{pmatrix}
        \begin{pmatrix}
            \star & \square & 0 \\
            0 & 0 & \star \\
            0 & 0 & 0
        \end{pmatrix}
    $} &

\textbf{Systems of Linear Equations}:
    Construct the \emph{augmented matrix}, use row operations to transform it
    into an Echelon form, reinterpret as a system of linear equations, and solve
    for each component in terms of a parameter. &

\textbf{Inverse (2x2)}:
    \smash{$\displaystyle
        A = \begin{pmatrix}
            a & b \\
            c & d
        \end{pmatrix},\,
        A^{-1} = \frac{1}{\det A}
        \begin{pmatrix}
            d & -b \\
            -c & a
        \end{pmatrix}
    $} \\

\hline

\textbf{Determinant}:
    In general, for a matrix $ A $, $ \det A = \sum_{i=1}^n (-1)^{i+j} a_{ij}
    \det(A_{ij}) $. For an upper triangular matrix $ A^E $ with diagonal $
    \mu_0, \ldots, \mu_n $, $ \det\left(A^E\right) = \prod_{i=1}^n \mu_i $. &

\textbf{Computing an Eigensystem (1)}:
    For the \emph{eigenvalues} of a matrix $ A $, solve $ \det(A-\lambda
    I_n) = 0 $ for $ \lambda $ (the eigenvalues), where $ I_n $ is the identity
    matrix in $ \mathbb{F}^{n \times n} $. &

\textbf{Computing an Eigensystem (2)}:
    For each eigenvalue $ \lambda_0 $, the corresponding \emph{eigenvector} $
    \vec{v}_0/C_0 $ is such that $ (A - \lambda_0 I_n) \vec{v}_0 = \vec{0} $,
    where $ C_0 \in \mathbb{F} $ is a nonzero constant. \\

\hline

\textbf{Eigendiagonalisation (1)}:
    If $ A \in \mathbb{F}^{n \times n} $ has linearly independent eigenvectors $
    A\vec{x}_1 = \lambda_1\vec{x}_1, \ldots, A\vec{x}_n = \lambda_n\vec{x}_n $,
    then $ A $ is similar to the diagonal matrix $ \Lambda $ with $ \lambda_0,
    \ldots, \lambda_n $. &

\textbf{Eigendiagonalisation (2)}:
    If the columns of a matrix $ M $ are given by the eigenvectors of $ A $,
    then $ AM = M\Lambda $, and $ \Lambda = M^{-1}AM $. &

\textbf{Eigendiagonalisation (3)}:
    If we have $ A = M \Lambda M^{-1} $ with matrices $ M $ invertible and $
    \Lambda $ diagonal, then $ A^n = (M \Lambda M^{-1})(M \Lambda M^{-1}) \ldots
    (M \Lambda M^{-1}) = M \Lambda^n M^{-1} $. \\

\hline
\end{tabular}

\vfill
\emph{Note:} Unless specified otherwise, an arbitrary point $ X $ (uppercase
Latin character) has position vector $ \vec{x} $ (lowercase), where $ \vec{x} =
(x_1, x_2, x_3) $.
\vfill

\clearpage
\begin{tabular}{|m{.31\linewidth}|m{.31\linewidth}|m{.31\linewidth}|}
\hline

\textbf{Norm of Matrix}:
    For a matrix $ A \in \mathbb{R}^{n \times n} $, its Euclidean norm is
    defined as $ \norm{A} = \max_{\norm{\vec{v}} \leq 1} \norm{A \vec{v}} $,
    where $ \norm{\vec{v}} $ is the Euclidean norm of $ \vec{v} $ in $
    \mathbb{R}^n $. &

\textbf{Exponential of a Matrix}:
    For some $ A \in \mathbb{R}^{n \times n} $, $ \exp A = \sum_{k=0}^\infty
    A^k/k! $. This sum is convergent, such that $ \norm{\exp A} = \exp \norm{A}
    $. If $ \dot{\vec{x}} = A\vec{x} $, then the general solution is given by $
    \vec{x}(t) = \exp(tA)\vec{v} $. &

\textbf{CODEs}:
    \smash{$
        \ldots\vec{x}(t) = \mathconst{e}^{tA}\vec{v} = M
        \begin{pmatrix}
            \mathconst{e}^{\lambda_1 t} & 0 & \ldots \\
            0 & \mathconst{e}^{\lambda_2 t} & \ldots \\
            \ldots & \ldots & \mathconst{e}^{\lambda_n t}
        \end{pmatrix}
        M^{-1}\vec{v}
    $} \\

\hline

\textbf{Real Symmetric Matrices}:
    Let $ A \in \mathbb{R}^{n \times n} $ with $ A^T = A $. Then, the
    eigenvalues of $ A $ are real, and eigenvectors belonging to distinct
    eigenvalues are orthogonal. &

\textbf{Orthogonal Matrices}:
    A square matrix $ M \in \mathbb{F}^{n \times n} $ satisfying $ M^T M = I_n $
    is an \emph{orthogonal} matrix. The columns of such a matrix are mutually
    orthonormal vectors. &

\textbf{Row Reduction Methods}:
    When finding the Echelon form, we can multiply rows by non-zero scalar
    constants, take the sum of two rows, and interchange rows. \\

\hline
\end{tabular}
\end{document}

